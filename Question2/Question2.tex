\documentclass[]{article}
\usepackage{lmodern}
\usepackage{amssymb,amsmath}
\usepackage{ifxetex,ifluatex}
\usepackage{fixltx2e} % provides \textsubscript
\ifnum 0\ifxetex 1\fi\ifluatex 1\fi=0 % if pdftex
  \usepackage[T1]{fontenc}
  \usepackage[utf8]{inputenc}
\else % if luatex or xelatex
  \ifxetex
    \usepackage{mathspec}
  \else
    \usepackage{fontspec}
  \fi
  \defaultfontfeatures{Ligatures=TeX,Scale=MatchLowercase}
\fi
% use upquote if available, for straight quotes in verbatim environments
\IfFileExists{upquote.sty}{\usepackage{upquote}}{}
% use microtype if available
\IfFileExists{microtype.sty}{%
\usepackage{microtype}
\UseMicrotypeSet[protrusion]{basicmath} % disable protrusion for tt fonts
}{}
\usepackage[margin=1in]{geometry}
\usepackage{hyperref}
\hypersetup{unicode=true,
            pdftitle={Yewno-Quantitative Analyst Question 2},
            pdfauthor={Leo Guoyuan Liu},
            pdfborder={0 0 0},
            breaklinks=true}
\urlstyle{same}  % don't use monospace font for urls
\usepackage{color}
\usepackage{fancyvrb}
\newcommand{\VerbBar}{|}
\newcommand{\VERB}{\Verb[commandchars=\\\{\}]}
\DefineVerbatimEnvironment{Highlighting}{Verbatim}{commandchars=\\\{\}}
% Add ',fontsize=\small' for more characters per line
\usepackage{framed}
\definecolor{shadecolor}{RGB}{248,248,248}
\newenvironment{Shaded}{\begin{snugshade}}{\end{snugshade}}
\newcommand{\KeywordTok}[1]{\textcolor[rgb]{0.13,0.29,0.53}{\textbf{{#1}}}}
\newcommand{\DataTypeTok}[1]{\textcolor[rgb]{0.13,0.29,0.53}{{#1}}}
\newcommand{\DecValTok}[1]{\textcolor[rgb]{0.00,0.00,0.81}{{#1}}}
\newcommand{\BaseNTok}[1]{\textcolor[rgb]{0.00,0.00,0.81}{{#1}}}
\newcommand{\FloatTok}[1]{\textcolor[rgb]{0.00,0.00,0.81}{{#1}}}
\newcommand{\ConstantTok}[1]{\textcolor[rgb]{0.00,0.00,0.00}{{#1}}}
\newcommand{\CharTok}[1]{\textcolor[rgb]{0.31,0.60,0.02}{{#1}}}
\newcommand{\SpecialCharTok}[1]{\textcolor[rgb]{0.00,0.00,0.00}{{#1}}}
\newcommand{\StringTok}[1]{\textcolor[rgb]{0.31,0.60,0.02}{{#1}}}
\newcommand{\VerbatimStringTok}[1]{\textcolor[rgb]{0.31,0.60,0.02}{{#1}}}
\newcommand{\SpecialStringTok}[1]{\textcolor[rgb]{0.31,0.60,0.02}{{#1}}}
\newcommand{\ImportTok}[1]{{#1}}
\newcommand{\CommentTok}[1]{\textcolor[rgb]{0.56,0.35,0.01}{\textit{{#1}}}}
\newcommand{\DocumentationTok}[1]{\textcolor[rgb]{0.56,0.35,0.01}{\textbf{\textit{{#1}}}}}
\newcommand{\AnnotationTok}[1]{\textcolor[rgb]{0.56,0.35,0.01}{\textbf{\textit{{#1}}}}}
\newcommand{\CommentVarTok}[1]{\textcolor[rgb]{0.56,0.35,0.01}{\textbf{\textit{{#1}}}}}
\newcommand{\OtherTok}[1]{\textcolor[rgb]{0.56,0.35,0.01}{{#1}}}
\newcommand{\FunctionTok}[1]{\textcolor[rgb]{0.00,0.00,0.00}{{#1}}}
\newcommand{\VariableTok}[1]{\textcolor[rgb]{0.00,0.00,0.00}{{#1}}}
\newcommand{\ControlFlowTok}[1]{\textcolor[rgb]{0.13,0.29,0.53}{\textbf{{#1}}}}
\newcommand{\OperatorTok}[1]{\textcolor[rgb]{0.81,0.36,0.00}{\textbf{{#1}}}}
\newcommand{\BuiltInTok}[1]{{#1}}
\newcommand{\ExtensionTok}[1]{{#1}}
\newcommand{\PreprocessorTok}[1]{\textcolor[rgb]{0.56,0.35,0.01}{\textit{{#1}}}}
\newcommand{\AttributeTok}[1]{\textcolor[rgb]{0.77,0.63,0.00}{{#1}}}
\newcommand{\RegionMarkerTok}[1]{{#1}}
\newcommand{\InformationTok}[1]{\textcolor[rgb]{0.56,0.35,0.01}{\textbf{\textit{{#1}}}}}
\newcommand{\WarningTok}[1]{\textcolor[rgb]{0.56,0.35,0.01}{\textbf{\textit{{#1}}}}}
\newcommand{\AlertTok}[1]{\textcolor[rgb]{0.94,0.16,0.16}{{#1}}}
\newcommand{\ErrorTok}[1]{\textcolor[rgb]{0.64,0.00,0.00}{\textbf{{#1}}}}
\newcommand{\NormalTok}[1]{{#1}}
\usepackage{graphicx,grffile}
\makeatletter
\def\maxwidth{\ifdim\Gin@nat@width>\linewidth\linewidth\else\Gin@nat@width\fi}
\def\maxheight{\ifdim\Gin@nat@height>\textheight\textheight\else\Gin@nat@height\fi}
\makeatother
% Scale images if necessary, so that they will not overflow the page
% margins by default, and it is still possible to overwrite the defaults
% using explicit options in \includegraphics[width, height, ...]{}
\setkeys{Gin}{width=\maxwidth,height=\maxheight,keepaspectratio}
\IfFileExists{parskip.sty}{%
\usepackage{parskip}
}{% else
\setlength{\parindent}{0pt}
\setlength{\parskip}{6pt plus 2pt minus 1pt}
}
\setlength{\emergencystretch}{3em}  % prevent overfull lines
\providecommand{\tightlist}{%
  \setlength{\itemsep}{0pt}\setlength{\parskip}{0pt}}
\setcounter{secnumdepth}{0}
% Redefines (sub)paragraphs to behave more like sections
\ifx\paragraph\undefined\else
\let\oldparagraph\paragraph
\renewcommand{\paragraph}[1]{\oldparagraph{#1}\mbox{}}
\fi
\ifx\subparagraph\undefined\else
\let\oldsubparagraph\subparagraph
\renewcommand{\subparagraph}[1]{\oldsubparagraph{#1}\mbox{}}
\fi

%%% Use protect on footnotes to avoid problems with footnotes in titles
\let\rmarkdownfootnote\footnote%
\def\footnote{\protect\rmarkdownfootnote}

%%% Change title format to be more compact
\usepackage{titling}

% Create subtitle command for use in maketitle
\newcommand{\subtitle}[1]{
  \posttitle{
    \begin{center}\large#1\end{center}
    }
}

\setlength{\droptitle}{-2em}
  \title{Yewno-Quantitative Analyst Question 2}
  \pretitle{\vspace{\droptitle}\centering\huge}
  \posttitle{\par}
  \author{Leo Guoyuan Liu}
  \preauthor{\centering\large\emph}
  \postauthor{\par}
  \predate{\centering\large\emph}
  \postdate{\par}
  \date{May 21, 2018}


\begin{document}
\maketitle

\textbf{Question 2} Implement one Smart Beta strategy and discuss pros
and cons compared to a chosen benchmark.

Traditional market-cap weighted portfolios have been criticized that
they are poorly diversified. The risk-based smart beta approach
corresponds to the criticism, and adopts risk-based weighing in
portfolio construction, thus manages the risk more effectively and
achieve a better performance. The well-known risk-based approaches are
MV (minimum variance), equally-weighted, and ERC (equally-weighted risk
contributions) In this exercise, I implement the ERC smart beta strategy
in R and compare its performance to other two as benchmarks.

\paragraph{Definition of ERC problem}\label{definition-of-erc-problem}

Given a portfolio \(x=(x_1,x_2,...,x_n)\) of n risky assets, \(x_i\) is
the weight of assets \(i\). The corresponding return is
\(r=(r_1,r_2,...,r_n)\). The goal of the ERC strategy is to find a
risk-balanced portfolio such that the risk contribution is the same for
all assets without short selling.

Let \(\Sigma=[\sigma_{ij}]\) be the covariance matrix of the return. The
portfolio risk is given by \(x^T \Sigma x\). The contribution of
\(i^th\) asset to the portfolio risk can be calculated by the weight
\(x_i\) multiply marginal risk contribution.
\[x_i\times \partial_{x_i} (x^T\Sigma x)=x_i (\Sigma x)_i \]

The problem can be rewritten as finding \(\hat{x}\) such that

\[\hat{x}=\{ x\in [0,1]^n:1^T x=1,x_i(\Sigma x)_i= x_j(\Sigma x)_j, \\
\text{ for all } i,j\} \]

We can use SQP ( Sequential Quadratic programming) to solve it

\[\begin{equation}
\begin{split}
\text{minimize } \,\displaystyle f(x)=\sum_{i,j}(x_i(\Sigma x)_i - x_j(\Sigma x)_j )^2\\
\text{subject to: } \displaystyle 1^Tx=1 \,, 0\le x\le1
\end{split}
\end{equation}\]

The function that calculates the optimized weights is \emph{erc}.

\paragraph{Comparison with 1/n and minimum-variance
portfolios}\label{comparison-with-1n-and-minimum-variance-portfolios}

Equally weighted portfolio(1/n) and minimum-variance (MV) portfolios are
widely used in practice. ERC portfolios are naturally located between
both and thus appear as good potential substitutes for these traditional
approaches Literature shown that

\begin{itemize}
\tightlist
\item
  The volatilities of these three portfolios as follows, with the ERC
  located between.
\end{itemize}

\[V_{mv}\le V_{erc}\le V_{1/n}\]

\begin{itemize}
\tightlist
\item
  ERC portfolio is optimal if we assume a constant correlation matrix
  and supposing that the assets have all the same Sharpe ratio.
\end{itemize}

\begin{Shaded}
\begin{Highlighting}[]
\CommentTok{# calulate the center quardratic moment}
\NormalTok{M2<-function (r) }
\NormalTok{\{}
    \NormalTok{L <-}\StringTok{ }\KeywordTok{nrow}\NormalTok{(r)}
    \NormalTok{rc <-}\StringTok{ }\KeywordTok{apply}\NormalTok{(r, }\DecValTok{2}\NormalTok{, scale, }\DataTypeTok{scale =} \OtherTok{FALSE}\NormalTok{)}
    \NormalTok{ans <-}\StringTok{ }\DecValTok{1}\NormalTok{/(L -}\StringTok{ }\DecValTok{1}\NormalTok{) *}\StringTok{ }\KeywordTok{crossprod}\NormalTok{(rc)}
    \NormalTok{ans}
\NormalTok{\}}
\CommentTok{# Portfoio performance}
\NormalTok{performance<-function(r)\{}
  \KeywordTok{c}\NormalTok{(}
  \DataTypeTok{Return=}\KeywordTok{Return.cumulative}\NormalTok{(r),}
  \DataTypeTok{Sharpe=}\KeywordTok{SharpeRatio}\NormalTok{(r,}\DataTypeTok{FUN=}\StringTok{"StdDev"}\NormalTok{),}
  \DataTypeTok{VaR=}\KeywordTok{VaR}\NormalTok{(r,}\DataTypeTok{method=}\StringTok{"historical"}\NormalTok{)}
\NormalTok{)}
\NormalTok{\}}

\NormalTok{## Function calculates ERC distrituion}
\NormalTok{## with respect to higher moments}
\NormalTok{##}
\CommentTok{#'}
\CommentTok{#' @param start \textbackslash{}code\{numeric\}, a \textbackslash{}eqn\{(N x 1)\} vector of starting values.}
\CommentTok{#' @param returns \textbackslash{}code\{matrix\}, a \textbackslash{}eqn\{(T x N)\} matrix of asset returns.}
\CommentTok{#' @param sigma \textbackslash{}code\{numeric\}, a \textbackslash{}eqn\{(N x N)\} Covariance matrix'}
\CommentTok{#'}
\CommentTok{#' @return  code\{numeric\}, a \textbackslash{}eqn\{(N x 1)\} vector of weights}
\CommentTok{#'}
\NormalTok{erc<-}\StringTok{ }\NormalTok{function(}\DataTypeTok{returns=}\OtherTok{NULL}\NormalTok{,}\DataTypeTok{sigma=}\OtherTok{NULL}\NormalTok{)\{}
  \CommentTok{#covariance matrix}
  \NormalTok{if(}\KeywordTok{is.null}\NormalTok{(sigma))}
    \NormalTok{sigma<-}\KeywordTok{M2}\NormalTok{(returns)}
  \CommentTok{#objective function}
  \NormalTok{f<-function(x)\{}
    \NormalTok{pctb<-}\StringTok{ }\NormalTok{x *}\StringTok{ }\NormalTok{sigma %*%}\StringTok{ }\NormalTok{x /}\StringTok{ }\KeywordTok{c}\NormalTok{(}\KeywordTok{crossprod}\NormalTok{(x, sigma) %*%}\StringTok{ }\NormalTok{x)}
    \KeywordTok{var}\NormalTok{(pctb)}
  \NormalTok{\}}
  
  \CommentTok{#optimization}
  \NormalTok{m<-}\KeywordTok{ncol}\NormalTok{(sigma)}
  \NormalTok{opt <-}\KeywordTok{nlminb}\NormalTok{(}\DataTypeTok{start =} \KeywordTok{runif}\NormalTok{(m), }\DataTypeTok{objective =} \NormalTok{f, }\DataTypeTok{lower =} \KeywordTok{rep}\NormalTok{(}\DecValTok{0}\NormalTok{, m))}
  \CommentTok{#output weight}
  \NormalTok{w <-}\StringTok{ }\NormalTok{opt$par /}\StringTok{ }\KeywordTok{sum}\NormalTok{(opt$par)}
\NormalTok{\}}

\CommentTok{#calculate risk contribution}
\NormalTok{risk_contribution<-function(r,w)\{}
  \NormalTok{dm2<-}\StringTok{ }\NormalTok{(}\KeywordTok{M2}\NormalTok{(r) %*%}\StringTok{ }\NormalTok{w)}
  \NormalTok{mp2<-}\KeywordTok{c}\NormalTok{(}\KeywordTok{crossprod}\NormalTok{(w, }\KeywordTok{M2}\NormalTok{(r)) %*%}\StringTok{ }\NormalTok{w)}
  \NormalTok{w*dm2/mp2}
\NormalTok{\}}



\NormalTok{mv<-function(}\DataTypeTok{returns=}\OtherTok{NULL}\NormalTok{,}\DataTypeTok{sigma=}\OtherTok{NULL}\NormalTok{)\{}
  \NormalTok{if(}\KeywordTok{is.null}\NormalTok{(sigma))}
    \NormalTok{sigma<-}\KeywordTok{M2}\NormalTok{(returns)}
  \KeywordTok{require}\NormalTok{(}\StringTok{"NMOF"}\NormalTok{)}
  \NormalTok{res<-}\KeywordTok{minvar}\NormalTok{(sigma)}
  \KeywordTok{as.vector}\NormalTok{(res)}
  
\NormalTok{\}}

\CommentTok{#calculate risk contribution}
\NormalTok{risk_contribution<-function(r,w)\{}
  \NormalTok{dm2<-}\StringTok{ }\NormalTok{(}\KeywordTok{M2}\NormalTok{(r) %*%}\StringTok{ }\NormalTok{w)}
  \NormalTok{mp2<-}\KeywordTok{c}\NormalTok{(}\KeywordTok{crossprod}\NormalTok{(w, }\KeywordTok{M2}\NormalTok{(r)) %*%}\StringTok{ }\NormalTok{w)}
  \NormalTok{w*dm2/mp2}
\NormalTok{\}}


\NormalTok{get_return<-function(x) x/}\KeywordTok{lag.xts}\NormalTok{(x)-}\DecValTok{1}
\end{Highlighting}
\end{Shaded}

\paragraph{A Numerical example}\label{a-numerical-example}

A numerical example is illustrated as follows. The portfolio has four
assets with constant volatility \((0.1,0.2,0.3,0.4)\) and correlation
matrix rho. Then 1/n portfolio give equal contribution of
\((0.25,0.25,0.25,0.25)\), while the mv portfolio is concentrated in the
first portfolio. The ERC portfolio is invested in all assets. Therefore
ERC portfolio is more balanced in terms of risk contribution. The
variance of the ERC portfolio is also between the other two.

\begin{Shaded}
\begin{Highlighting}[]
\NormalTok{rho<-}\KeywordTok{matrix}\NormalTok{(}
  \KeywordTok{c}\NormalTok{(}\DecValTok{1}\NormalTok{,}\FloatTok{0.8}\NormalTok{,}\DecValTok{0}\NormalTok{,}\DecValTok{0}\NormalTok{, }\FloatTok{0.8}\NormalTok{,}\DecValTok{1}\NormalTok{,}\DecValTok{0}\NormalTok{,}\DecValTok{0}\NormalTok{,}\DecValTok{0}\NormalTok{,}\DecValTok{0}\NormalTok{,}\DecValTok{1}\NormalTok{,-}\FloatTok{0.5}\NormalTok{,}\DecValTok{0}\NormalTok{,}\DecValTok{0}\NormalTok{,-}\FloatTok{0.5}\NormalTok{,}\DecValTok{1}\NormalTok{),}
  \KeywordTok{c}\NormalTok{(}\DecValTok{4}\NormalTok{,}\DecValTok{4}\NormalTok{)}
\NormalTok{)}
\NormalTok{s<-}\KeywordTok{c}\NormalTok{(}\FloatTok{0.1}\NormalTok{,}\FloatTok{0.2}\NormalTok{,}\FloatTok{0.3}\NormalTok{,}\FloatTok{0.4}\NormalTok{)}
\NormalTok{d<-}\KeywordTok{diag}\NormalTok{(s)}
\KeywordTok{print}\NormalTok{(}\StringTok{"The correlation matrix portfolio"}\NormalTok{)}
\end{Highlighting}
\end{Shaded}

\begin{verbatim}
[1] "The correlation matrix portfolio"
\end{verbatim}

\begin{Shaded}
\begin{Highlighting}[]
\KeywordTok{print}\NormalTok{(rho)}
\end{Highlighting}
\end{Shaded}

\begin{verbatim}
     [,1] [,2] [,3] [,4]
[1,]  1.0  0.8  0.0  0.0
[2,]  0.8  1.0  0.0  0.0
[3,]  0.0  0.0  1.0 -0.5
[4,]  0.0  0.0 -0.5  1.0
\end{verbatim}

\begin{Shaded}
\begin{Highlighting}[]
\KeywordTok{print}\NormalTok{(}\StringTok{"The stdv of the portfolio"}\NormalTok{)}
\end{Highlighting}
\end{Shaded}

\begin{verbatim}
[1] "The stdv of the portfolio"
\end{verbatim}

\begin{Shaded}
\begin{Highlighting}[]
\KeywordTok{print}\NormalTok{(s)}
\end{Highlighting}
\end{Shaded}

\begin{verbatim}
[1] 0.1 0.2 0.3 0.4
\end{verbatim}

\begin{Shaded}
\begin{Highlighting}[]
\NormalTok{Sigma<-d%*%rho%*%d}
\NormalTok{Vr<-function(w,}\DataTypeTok{sigma=}\NormalTok{Sigma)\{}
 \KeywordTok{as.numeric}\NormalTok{( }\KeywordTok{t}\NormalTok{(w)%*%sigma%*%w)}
\NormalTok{\}}
\KeywordTok{print}\NormalTok{(}\StringTok{"The solution for the 1/n portfolio"}\NormalTok{)}
\end{Highlighting}
\end{Shaded}

\begin{verbatim}
[1] "The solution for the 1/n portfolio"
\end{verbatim}

\begin{Shaded}
\begin{Highlighting}[]
\NormalTok{w=}\KeywordTok{rep}\NormalTok{(}\DecValTok{1}\NormalTok{/}\DecValTok{4}\NormalTok{,}\DecValTok{4}\NormalTok{)}
\KeywordTok{print}\NormalTok{(w )}
\end{Highlighting}
\end{Shaded}

\begin{verbatim}
[1] 0.25 0.25 0.25 0.25
\end{verbatim}

\begin{Shaded}
\begin{Highlighting}[]
\KeywordTok{print}\NormalTok{(}\StringTok{"The variance"}\NormalTok{)}
\end{Highlighting}
\end{Shaded}

\begin{verbatim}
[1] "The variance"
\end{verbatim}

\begin{Shaded}
\begin{Highlighting}[]
\KeywordTok{print}\NormalTok{(}\KeywordTok{Vr}\NormalTok{(w))}
\end{Highlighting}
\end{Shaded}

\begin{verbatim}
[1] 0.01325
\end{verbatim}

\begin{Shaded}
\begin{Highlighting}[]
\KeywordTok{print}\NormalTok{(}\StringTok{"The solution for the MV portfolio"}\NormalTok{)}
\end{Highlighting}
\end{Shaded}

\begin{verbatim}
[1] "The solution for the MV portfolio"
\end{verbatim}

\begin{Shaded}
\begin{Highlighting}[]
\NormalTok{w=}\KeywordTok{mv}\NormalTok{(}\DataTypeTok{sigma=}\NormalTok{Sigma)}
\KeywordTok{print}\NormalTok{(w)}
\end{Highlighting}
\end{Shaded}

\begin{verbatim}
[1] 0.7448276 0.0000000 0.1517241 0.1034483
\end{verbatim}

\begin{Shaded}
\begin{Highlighting}[]
\KeywordTok{print}\NormalTok{(}\StringTok{"The variance"}\NormalTok{)}
\end{Highlighting}
\end{Shaded}

\begin{verbatim}
[1] "The variance"
\end{verbatim}

\begin{Shaded}
\begin{Highlighting}[]
\KeywordTok{print}\NormalTok{(}\KeywordTok{Vr}\NormalTok{(w))}
\end{Highlighting}
\end{Shaded}

\begin{verbatim}
[1] 0.007448276
\end{verbatim}

\begin{Shaded}
\begin{Highlighting}[]
\KeywordTok{print}\NormalTok{(}\StringTok{"The solution for the ERC portfolio"}\NormalTok{)}
\end{Highlighting}
\end{Shaded}

\begin{verbatim}
[1] "The solution for the ERC portfolio"
\end{verbatim}

\begin{Shaded}
\begin{Highlighting}[]
\NormalTok{w=}\KeywordTok{erc}\NormalTok{(}\DataTypeTok{sigma=}\NormalTok{Sigma)}
\KeywordTok{print}\NormalTok{(w )}
\end{Highlighting}
\end{Shaded}

\begin{verbatim}
[1] 0.5135543 0.2567772 0.2296685 0.0000000
\end{verbatim}

\begin{Shaded}
\begin{Highlighting}[]
\KeywordTok{print}\NormalTok{(}\StringTok{"The variance"}\NormalTok{)}
\end{Highlighting}
\end{Shaded}

\begin{verbatim}
[1] "The variance"
\end{verbatim}

\begin{Shaded}
\begin{Highlighting}[]
\KeywordTok{print}\NormalTok{(}\KeywordTok{Vr}\NormalTok{(w))}
\end{Highlighting}
\end{Shaded}

\begin{verbatim}
[1] 0.01424186
\end{verbatim}

\paragraph{Implement the ERC
strategy.}\label{implement-the-erc-strategy.}

I write an Run\_Strategy function to implement the three strategies. The
idea behind the strategy is that we rebalance the portfolio every 30
days based on the weights calculated by the strategy. We using the
rolling window of 360 days to calculate the variance matrix. The
performance function measure the performance of the strategies,
including the cumulative return, the Sharpe ratio and the VaR. The
performance of the all three metrics are between 1/n and MV.

\begin{Shaded}
\begin{Highlighting}[]
\NormalTok{data <-}\StringTok{ }\KeywordTok{new.env}\NormalTok{()}
\KeywordTok{getSymbols}\NormalTok{(}\KeywordTok{c}\NormalTok{(}\StringTok{'AAPL'}\NormalTok{,}\StringTok{'WMT'}\NormalTok{,}\StringTok{'GE'}\NormalTok{,}\StringTok{'IBM'}\NormalTok{), }\DataTypeTok{src =} \StringTok{'yahoo'}\NormalTok{,}\DataTypeTok{from =} \StringTok{"2000-01-01"}\NormalTok{, }\DataTypeTok{adjust =}  \OtherTok{TRUE}\NormalTok{,}\DataTypeTok{env=}\NormalTok{data )}
\end{Highlighting}
\end{Shaded}

\begin{verbatim}
[1] "AAPL" "WMT"  "GE"   "IBM" 
\end{verbatim}

\begin{Shaded}
\begin{Highlighting}[]
\NormalTok{x<-}\KeywordTok{merge}\NormalTok{(data$AAPL[,}\DecValTok{6}\NormalTok{], data$WMT[,}\DecValTok{6}\NormalTok{], data$GE[,}\DecValTok{6}\NormalTok{],data$IBM[,}\DecValTok{6}\NormalTok{])}

\NormalTok{## Function runs the rebalance strategy E}
\NormalTok{## with respect to higher moments}
\NormalTok{##}
\CommentTok{#'}
\CommentTok{#' @param x market price data.}
\CommentTok{#' @param stragegy name of the strategy namely 1/n, mv, and erc#'}
\CommentTok{#'}
\CommentTok{#' @return  return of the portfolio}
\CommentTok{#'}
\NormalTok{Run_Strategy<-function(x,strategy)\{}
  \NormalTok{n=}\KeywordTok{nrow}\NormalTok{(x)}
  \NormalTok{m =}\KeywordTok{ncol}\NormalTok{(x)}
  \NormalTok{r=}\KeywordTok{get_return}\NormalTok{(x)}
  
  
  
  \NormalTok{res<-}\KeywordTok{xts}\NormalTok{(}\KeywordTok{data.frame}\NormalTok{(}\DataTypeTok{Value=}\KeywordTok{rep}\NormalTok{(}\DecValTok{1}\NormalTok{,n)),}\DataTypeTok{order.by=}\KeywordTok{index}\NormalTok{(x))}
  \CommentTok{#starting from equal weights}
  \NormalTok{shares=}\KeywordTok{rep}\NormalTok{(}\DecValTok{1}\NormalTok{/m, m)/}\KeywordTok{as.numeric}\NormalTok{(x[}\DecValTok{360}\NormalTok{])}

  \NormalTok{for(i in }\DecValTok{361}\NormalTok{:n)\{}
     \NormalTok{res[i]$Value=}\KeywordTok{sum}\NormalTok{(x[i]*shares)}
     \CommentTok{# re-balance the portfolio every 30 days}
    \NormalTok{if((i}\DecValTok{-361}\NormalTok{)%%}\DecValTok{30}\NormalTok{==}\DecValTok{0}\NormalTok{)\{}
      \CommentTok{# switch strategy}
      \NormalTok{sigma=}\KeywordTok{M2}\NormalTok{(r[(i}\DecValTok{-360+1}\NormalTok{):i])}
      \NormalTok{if(strategy==}\StringTok{"1/n"}\NormalTok{)\{}
        \NormalTok{w=}\KeywordTok{rep}\NormalTok{(}\DecValTok{1}\NormalTok{/m,m)}
      \NormalTok{\}else if(strategy==}\StringTok{"mv"}\NormalTok{)\{}
        \NormalTok{w=}\KeywordTok{mv}\NormalTok{(}\DataTypeTok{sigma=}\NormalTok{sigma)}
      \NormalTok{\}else if(strategy==}\StringTok{"erc"}\NormalTok{)\{}
         \NormalTok{w=}\KeywordTok{erc}\NormalTok{(}\DataTypeTok{sigma=}\NormalTok{sigma)}
      \NormalTok{\}else\{}
        \KeywordTok{stop}\NormalTok{(}\StringTok{"not implented"}\NormalTok{)}
      \NormalTok{\}}
     
      \NormalTok{shares=}\KeywordTok{as.numeric}\NormalTok{(res[i]$Value*w)/}\KeywordTok{as.numeric}\NormalTok{(x[i])}
    \NormalTok{\}}
   
  \NormalTok{\}}
  \CommentTok{# get return}
  \KeywordTok{get_return}\NormalTok{(res)[}\StringTok{"2002/"}\NormalTok{]}
\NormalTok{\}}



\NormalTok{r1<-}\KeywordTok{Run_Strategy}\NormalTok{(x,}\StringTok{"1/n"}\NormalTok{)}
\NormalTok{r2<-}\KeywordTok{Run_Strategy}\NormalTok{(x,}\StringTok{"mv"}\NormalTok{)}
\NormalTok{r3<-}\KeywordTok{Run_Strategy}\NormalTok{(x,}\StringTok{"erc"}\NormalTok{)}


\KeywordTok{performance}\NormalTok{(r1)}
\end{Highlighting}
\end{Shaded}

\begin{verbatim}
     Return      Sharpe         VaR 
 5.06867194  0.04111114 -0.01960185 
\end{verbatim}

\begin{Shaded}
\begin{Highlighting}[]
\KeywordTok{performance}\NormalTok{(r2)}
\end{Highlighting}
\end{Shaded}

\begin{verbatim}
     Return      Sharpe         VaR 
 1.18994752  0.02273673 -0.01709234 
\end{verbatim}

\begin{Shaded}
\begin{Highlighting}[]
\KeywordTok{performance}\NormalTok{(r3)}
\end{Highlighting}
\end{Shaded}

\begin{verbatim}
     Return      Sharpe         VaR 
 3.04278114  0.03463334 -0.01852141 
\end{verbatim}

\paragraph{Conclusion}\label{conclusion}

Minimum variance (MV) and equality-weighted (1/n) portfolio are popular
risk-based smart beta strategy. They both have limitations. 1/n lacks
risk monitoring while mv suffers dramatic asset concentration. The ERC
strategy offer a middle ground between both the distribution and the
risk. From the implementation of the simple rebalance strategy, the
performance of the ERC is also between the other two, thus give a
meaningful option for portfolio construciton.

\textbf{Reference} On the properties of equally-weighted risk
contributions portfolios Sébastien Maillard, Thierry Roncalli and Jérôme
Teiletche May 2009
\url{https://papers.ssrn.com/sol3/papers.cfm?abstract_id=1271972}


\end{document}
